\documentclass[14pt]{scrarticle}
\usepackage[english,russian]{babel}
\usepackage{amsmath}
\linespread{1.3}
\usepackage{geometry}
\geometry{
        left=3.5cm,
        right=1.5cm,
        top=2cm,
        bottom=1.5cm,
}   

\begin{document}
\begin{titlepage}
\begin{center}
    ОТЗЫВ Научного руководителя\\
    на магистерскую диссертацию\\ студента 2-го курса магистратуры Физического факультета Национального Университета Узбекистана им. Мирзо Улугбека \\по направлению "Ядерная физика и ядерные технологии"\\

    Ашурова Синдоржона Ахмаджон угли\\

    "Исследование развал ядер кислорода на $\alpha$ частицу и на $^{12}C$ в $^{16}$Op соударениях при 3,25 А ГэВ/с"
\end{center}

Представленная диссертация посвящена теоретическому исследованию кластеризацию легких ядер методом ядерных эмульсий. Тема работы является весьма актуальной в связи с развитием физиики высоких энергий и других научно-практических задач,пос-кольку данные материалы используются для изучении кластерной структуры ядра. Диссертационная работа состоит из введения, трех глав, заключения и двух приложений, всего 68 страниц. Список литературы содержит 57 наименования. 


Сама работа является весьма сбалансированной, написана грамотно, хорошим языком. В работе представлен большой обзор имеющихся как экспериментальных, так и теоретических данных, проведено их сопоставления с результатами, полученными в данной работе. Также выполнен подробный анализ полученных результатов: периферические соударении ядра кислорода и протона, сечение реакции, число образовавщихся фрагментов.

Следует отметить, что результаты работы докладывались на ряде конференций, а также по результатам работы были подготовлены две публикации, которые уже приняты в печать. 

Считаю,   что   работа соответствует уровню магистерской диссертации, а соискатель, Ашуров Синдоржон, степени магистра ядерная физика и ядерные технологии. Рекомендуемая оценка: "5"(отлично).
\vfill

\noindent Научный руководитель:\\
АН РУз ИЯФ \\\
д.ф.-м.н., проф.\ \ \ \ \ \ \  \ \ \ \ \ \ \ \ \ \ \ \ \ \ \ \ \ \ \ \ \ \ \ \ \ \ \ \ \ \ \ \ \ \ \ \ \ \  Бозоров Э.Х.

\vfill
\end{titlepage}
\end{document}
