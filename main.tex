\documentclass[fontsize=14pt]{scrarticle}
\usepackage{amssymb}
\usepackage[english,russian]{babel}
\usepackage{tikz}
\linespread{1.6}
\usepackage{xcolor}
\usepackage{hyperref}
\usepackage{multirow,tabularx}
\newcolumntype{Y}{>{\centering\arraybackslash}X}
\renewcommand{\arraystretch}{1.3}
\usepackage[nottoc,notlof,notlot]{tocbibind} 
\usepackage{comment}
\usepackage{geometry}
\usepackage{amsmath}
\usepackage{graphicx}
\usepackage{nomencl}
\usepackage{fancyhdr}
\usepackage{caption}
\usepackage{subcaption}
\usepackage{hyperref}
\usepackage{comment}
\graphicspath{{pictures/}}
\DeclareGraphicsExtensions{.pdf,.png,.jpg}
\linespread{1.6}
\geometry{
	paper=a4paper,
	top=3.5cm,
	bottom=2.5cm,
	right=2cm,
	left=3.5cm
}
\begin{document}
	\begin{titlepage}
		\begin{center}
		\large	Министерство высшего и среднего специального образования Республики Узбекистан 
Национальный  Университет Узбекистана  
Физический Факультет Кафедра Ядерной физики\\
 5A140207 Ядерная физика и ядерные технологии
			\vspace{0.25cm}
			
			
			\begin{flushright}
			\small{На правах рукописи}\\
           \small{ УДК 621.039.546.8}
			\end{flushright}
			\vfill
			
			\large \textbf{МАГИСТЕРСКАЯ ДИССТЕРТАЦИЯ }
			
			Ашуров Синдор
			\vfill
			{\large Исследование развал ядра кислорода на альфа-частицу и $^{12}$C в $^{16}$Оp соударениях при 3,25 А ГэВ/с\\
			}
			\bigskip
		\end{center}

         
         \vfill
         
         \begin{flushleft}

Науч. руководитель:\_\_\_\_\_\_\_ д.ф.-м.н, проф. БОЗОРОВ Э.Х.
         \end{flushleft}
		\vfill
		
		\vfill
		\begin{center}
			г. Ташкент-2021
		\end{center}
	\end{titlepage}
	\newpage
	\makenomenclature 
	\renewcommand{\nomname}{Перечень условных обозначений}
    \newcommand*{\nom}[2]{#1\nomenclature{#1}{#2}}
	\printnomenclature[5em]
	\newpage
	\renewcommand{\contentsname}{Содержание}
	\tableofcontents
	\newpage
	
\addcontentsline{toc}{section}{Введение}
\section*{Введение}
	Коллективные степени свободы, в которых группы из нескольких нуклонов ведут себя как составляющие кластеры, являются одним из ключевых аспектов ядерной структуры. Основными «строительными кирпичами» кластеризации являются легчайшие ядра, не имеющие возбужденных состояний – прежде всего это ядра 4-He ($\alpha$-частицы), а также дейтроны (d), тритоны (t) и ядра 3-He (гелионы). Эта особенность отчетливо проявляется в легких ядрах, где число возможных кластерных конфигураций невелико (рис.1). В частности, пороги отделения кластеров в ядрах 7-Be, 6,7-Li, 11,10-B, 11,12-C и 16-O оказываются ниже порогов отделения нуклонов. Ярко выраженную кластерную природу имеют как стабильное ядро 9 Be, так и несвязанные ядра 8 Be и 9 B. Кластерные ядра 7 Be, 7 Li, 8 Be и 9 B выступают в качестве основ в изотопах 8 B и 9-12 C. Описания основных состояний легких ядер в оболочечной и кластерной модели является взаимодополняющим. В кластерной картине легкие ядра представляются как суперпозиции различных конфигураций кластеров и нуклонов. Интерес к таким состояниям связан с предсказанием их свойств, как молекулярно-подобных [1,2].
	
Кластеризация ядер традиционно рассматривается как прерогатива физики ядерных реакций низких энергий [3]. Целью настоящих лекционных заметок является представление потенциала одного из разделов физики высоких энергий – релятивистской ядерной физики – для развития концепций ядерной кластеризации. 

В последнее десятилетие получили развитие концепции разреженной ультрахолодной ядерной материи, основанной на конденсации нуклонов в легчайшие ядра [4–6]. Как аналог атомных квантовых газов рассматривается $\alpha$-частичный конденсат Бозе-Эйнштейна [5,7]. Эти разработки выдвигают проблему изучения разнообразных кластерных ансамблей и несвязанных ядер как фундаментальных компонент новых квантовых сред. Когерентные ансамбли кластеров в макроскопическом масштабе могут играть промежуточную роль в нуклеосинтезе, что придает изучению ядерной кластеризации значение, существенно выходящее за рамки проблем ядерной структуры. На первый взгляд исследования ядерных систем многих тел кажутся невозможными в лабораторных условиях. Тем не менее, они могут быть исследованы непрямым образом в процессах развала ядер при возбуждении несколько выше соответствующего порога. Конфигурационное перекрытие основного состояния фрагментирующего ядра с конечными кластерными состояниями полно проявляется при взаимодействиях на периферии ядра мишени, когда вносимое возмущение минимально. Представляется, что явление периферической диссоциации релятивистских ядер может служить альтернативной "лабораторией" для изучения беспрецедентного разнообразия кластерных ансамблей. В основе этой идеи лежат следующие факты. При энергии столкновения ядер свыше 1A ГэВ кинематические области фрагментации ядра снаряда и ядра мишени отчетливо разделяются, а импульсные спектры фрагментов выходят на асимптотическое поведение. Тем самым достигается режим предельной фрагментации ядер, что также означает неизменность изотопического состава фрагментов при возрастании энергии столкновения. Особую ценность для кластерной физики имеют события периферической диссоциации налетающего ядра с сохранением числа нуклонов в области его фрагментации. При энергии налетающих ядер свыше 1A ГэВ вероятность такой диссоциации достигает нескольких процентов. Определение взаимодействий как периферических упрощается благодаря возрастающей коллимации фрагментов. Пороги детектирования релятивистских фрагментов отсутствуют, а теряемая ими энергия в детекторах минимальна. Все эти обстоятельства принципиально важны для экспериментальных исследований.
\paragraph{Актуальность и востребованность темы диссертации} В настоящее время экспериментальные генерации частиц и исследования
фрагментации процессов релятивистских множественной ядер являются
чрезвычайно важными для решения фундаментальных вопросов физики
высоких энергий и релятивистской ядерной физики. Одним из важнейших
источников информации о структуре ядер и ее влиянии на состав конечных продуктов реакций, а также роли зарядовообменных процессов при фрагментации ядер является исследование соударений релятивистских ядер с нуклонами и ядрами в полуинклюзивных и максимально приближенных к эксклюзивным реакциях (в условиях 4$\pi$-геометрии,
с полной идентификацией фрагментов и измерением их импульсов и углов вылета). Таким требованиям к экспериментальным данным наиболее полно соответствуют эксперименты, выполняемые с помощью пузырьковых камер, экспонируемых в сильных магнитных полях в пучках релятивистских ядер.
\paragraph{Целью исследования}  является установление закономерностей образования многонуклонных систем и ядер с массовыми числами А$\le$7,
влияния исходной структуры ядра кислорода и зарядообменных процессов на состав и выходы конечных продуктов в 16 Ор-соударениях при 3.25 А ГэВ/с. В соответствии с поставленной целью в соударениях ядер кислорода с протонами при 3.25 А ГэВ/с необходимо было решить следующие задачи:
\begin{itemize}
    \item разработать и апробировать новый метод разделения протонов и положительно заряженных пионов, визуально не различимых в области импульсов 1.25 - 1.75 ГэВ/с, а также методику учета поправок на потерю различных типов частиц и фрагментов
    \item исследовать основные закономерности образования 6- и 7-нуклон-ных систем и ядер
    \item изучить средние множественности и кинематические характеристики различных частиц (протонов отдачи, нейтронов-фрагментов и заряженных пионов) и фрагментов с А$\le$ 4, сопровождающих образование 6- и 7- нуклонных систем и ядер
    \item изучить особенности и механизмы образования легких зеркальных ядер $^{3}$He, $^{3}$H, $^{7}$Li, $^{7}$Be и их корреляции с образованием различного числа дейтронов и $\alpha$-частиц
    \item выполнить сравнительный анализ экспериментальных результатов и предсказаний каскадно-фрагментационной испарительной модели с целью выявления и установления роли $\alpha$-кластерной структуры ядра кислорода в процессах его фрагментации.
\end{itemize}




\paragraph{Объектом исследования} являются процессы фрагментации, происходящие при столкновениях ядер кислорода с протонами при импульсе 3.25А ГэВ/с.
\paragraph{Предметом исследования}  являются многонуклонные системы и ядра, образующиеся во взаимодействиях ядер кислорода с протонами при 3.25 А ГэВ/с.
\paragraph{Методы исследования} Для решения поставленных задач по анализу образования многонуклонных систем и ядер в $^{16}$Ор-соударениях при 3.25 А ГэВ/с использован инклюзивный и полуинклюзивный подход с применением математической статистики и методов Монте - Карловского моделирования. 
\paragraph{Научная новизна} исследования заключаются в следующем:
\begin{itemize}
    \item впервые определены инклюзивные и полуинклюзивные сечения
образования 6- и 7-нуклонных систем и ядер в 16 Ор-соударениях при 3.25 А ГэВ/с
\item установлено, что средние множественности сопровождающих частиц
определяются в основном суммарным массовым числом и зарядом конечного
многонуклонного состояния и не зависят от того, является ли оно единым ядром или составным состоянием двух или трех ядер с тем же суммарным массовым числом А
\item впервые установлена независимость средних множественностей
протонов- и нейтронов-фрагментов от числа ассоциированных дейтронов,
указывающая на то, что основным механизмом образования дейтронов в
рассматриваемых каналах является разрушение $\alpha$-кластеров ядра кислорода
\end{itemize}

	
	
	
\section{Краткий обзор состояния теоретических и экспериментальных исследований фрагментации ядер}	
\subsection{Физика релятивистских ядер}	
\hspace{0.6cm}

Кластерные ансамбли, возникающие при фрагментации релятивистских ядер, наиболее полно наблюдаются в ядерной эмульсии. В качестве примера на рис. 2 представлена макрофотография взаимодействия в эмульсии ядра 28-Si с энергией 3.65 A ГэВ. Зернистость снимка составляет около 0.5 $\mu$м. Интерес представляет группа релятивистких фрагментов Н и He с суммарным зарядом $\Sigma Z_{fr}$ = 13. На верхней фотографии струя фрагментов в узком конусе, сопровождаемая четырьмя однозарядными релятивистскими частицами в широком конусе и тремя осколками ядра мишени. Со смещением по направлению струи фрагментов (нижняя фотография) можно различить три фрагмента H и 5 фрагментов He. Интенсивный "след" на нижней фотографии (третий сверху) раздваивается в пару следов с $Z_{fr}$fr = 2 и углом разлета около 2$\cdot$10$^{-3}$ рад, что соответствует распаду ядра 8-Be. Столь узкие распады достаточно часто наблюдаются при фрагментации релятивистских ядер. Они свидетельствуют о полноте
наблюдения по всему спектру кластерных возбуждений. 
	
	Ядерная кластеризация описывает появление молекулярных структур в ядерной физике. В молекулах существует богатая феноменология различных химических связей, сложных вращательных и колебательных возбуждений и сложной структурной геометрии. Может ли быть такой же уровень сложности в ядерных системах? Возможности, безусловно, существуют с сильной связью между четырьмя нуклонами в $\alpha$-частице и последствиями почти связанного динейтронного канала. Однако физика, лежащая в основе, усложняется демократией частиц, участвующих в ядерном связывании. Вместо тяжелых ионов, окруженных легкими электронами, протоны и нейтроны имеют примерно равные массы, и кластерные структуры возникают из тонкого баланса между короткодействующими отталкивающими силами и блокирующими эффектами Паули , средними ядерными силами притяжения и дальнодействующим кулоновским отталкиванием протонов. 
	
На самом деле изучение кластеризации ядер началось с открытия Резерфордом альфа-излучения (Резерфорд, 1899) и развития квантовой механики. Гамов (Gamow, 1928) и, независимо, Герни и Кондон (Gurney and Condon, 1928) описали $\alpha$-частицу как подвергающуюся квантово-механическому туннелированию изнутри распадающегося ядра. Примерно десять лет спустя Уиллер (Wheeler, 1937a) разработал метод резонирующих групп для описания $\alpha$-кластеров и других групп кластеров внутри ядер, позволяя протонам и нейтронам сохранять свою фермионную квантовую статистику. Затем вышли работу Hafstad и Теллер, в котором описанной четно-четные N = Z ядра в терминах  $\alpha$ ; - модели частиц с соединительными связями кластеров (Хафстад и Теллер, 1938). Следуя тому же пути, Деннсион предложил модель низколежащих состояний 16 O в терминах четырех $\alpha$-кластеров в вершинах правильного тетраэдра (Dennison, 1940, 1954). На более микроскопическом уровне Маргенау использовал детерминантную волновую функцию Слэтера для $\alpha$-кластеров, чтобы вычислить эффективное $\alpha$-$\alpha$ взаимодействие (Margenau, 1941). 

Спустя несколько лет Моринага предположил, что несферические и даже линейные цепочки $\alpha$-кластеров могут описывать некоторые состояния $\alpha$-подобных ядер (Morinaga, 1956). Одним из кандидатов на такое описание было второе состояние 0$^{+}$ 12 C, постулированное Хойлом (Hoyle, 1954) как ответственное за усиление тройной $\alpha$-реакции в звездах и экспериментально наблюдаемое вскоре после этого (Cook et al., 1957). Одновременно с этими теоретическими разработками новые эксперименты предоставили высококачественные данные об упругом $\alpha$-$\alpha$-рассеянии (Афзал и др., 1969; Гейденбург и Теммер, 1956; Нильсон и др., 1958). Это, в свою очередь, привело к развитию эффективного $\alpha$-$\alpha$ взаимодействия (Али и Бодмер, 1966). 

Примерно в то же время Бринк использовал детерминантную волновую функцию Марньо Слейтера для $\alpha$-кластера и метод координат генератора, чтобы упростить вычисления, которые были трудными в более общем формализме метода резонирующих групп (Brink, 1966a). Эквивалентность метода координат генератора и метода резонирующих групп была позже выяснена Хориучи (Horiuchi, 1970). Что касается $\alpha$-распадов, Кларк и Ван вычислили вероятность образования $\alpha$-кластеров вблизи поверхности тяжелых ядер (Clark and Wang, 1966). Между тем Икеда, Такигава и Хориучи заметили, что $\alpha$-кластеризация появилась близко к порогам $\alpha$-распада, и они были схематично обозначены так называемыми диаграммами Икеда (Ikeda et al., 1968). Следуя этим же концепциям, изучение кластеризации было расширено на богатые протонами и нейтронно-богатые системы с близкими открытыми порогами. В соответствующих состояниях являются слабо связанными системами кластеров и избыточных нейтронов или протонов. 

Был опубликован ряд обзоров кластеризации ядер (Akaishi et al., 1986; Beck, 2010, 2012, 2014; Freer, 2007; Funaki et al., 2015; Horiuchi et al., 2012; von Oertzen et al. , 2006). Из-за нехватки места невозможно подробно охватить все области исследований. Тем не менее, мы стараемся дать сбалансированный взгляд на поле с точки зрения команды практиков, владеющих целым рядом методов и опытом. В обзоре теоретических методов мы сосредотачиваемся на микроскопической кластеризации, когда кластеры возникают из нуклонных степеней свободы. Поскольку эта область динамична и развивается, некоторые ключевые вопросы в настоящее время не решены, и существуют разногласия между различными методами. Более того, некоторые из наиболее интересных результатов, вероятно, будут получены в ближайшем будущем. Этого следовало ожидать в растущей области, где многие активно занимаются важными открытыми вопросами и исследованиями. Полезно кратко суммировать сильные стороны и проблемы различных теоретических подходов. Большинство обсуждаемых нами методов представляют собой вариационные вычисления с использованием некоторого предписанного анзаца для ядерной волновой функции. Они включают в себя антисимметризованную молекулярную динамику, фермионную молекулярную динамику, в Tohsaki-Хориучи - волновая функция и модель контейнера Шук-Репка, и микроскопические модели кластера, используя резонирующую группу или генератор координат методов. Эти вариационные подходы часто дают хорошее согласие с экспериментальными данными, а также интуитивную картину лежащих в основе ядерных волновых функций. Основные задачи состоят в том, чтобы включить первые принципы ядерных сил и устранить систематические ошибки, связанные с выбором вариационных базисных состояний.  
Некоторые вариационные методы также были объединены с методами Монте-Карло. Вариационный Монте-Карло использует стохастическую выборку для вычисления интегралов перекрытия. Он также часто используется в качестве отправной точки для моделирования диффузии или моделирования Монте-Карло с помощью функции Грина. В этих расчетах использовались ядерные силы из первых принципов, и систематические ошибки можно оценить, допустив неограниченную эволюцию квантовой волновой функции. Основная задача для этих расчетов является то , что вычислительные затраты экспоненциально возрастает с увеличением числа частиц. Другой метод, называемый оболочечной моделью Монте-Карло, использует вспомогательное поле Монте-Карло для выбора оптимизированных состояний вариационного базиса. Как и в случае с другими вариационными методами, проблема заключается в систематических ошибках из-за выбора базисных состояний. 

Модель оболочки без ядра с расчетами континуума начинается с первых принципов ядерных сил, описываемых киральной эффективной теорией поля, и продемонстрировала впечатляющее согласие свойств континуума легких ядер. Подобно функции Грина Монте-Карло, проблема этого метода заключается в экспоненциальном масштабировании усилий при работе с более крупными системами. Модель оболочки без ядра, адаптированная к симметрии, предлагает несколько многообещающих идей для эффективной мобилизации вычислительных ресурсов на основе симметрии. Тем не менее остаются трудности в достижении более крупных систем с использованием ядерных сил из первых принципов.

Теория эффективного поля ядерной решетки использует киральную теорию эффективного поля и методы решеточного Монте-Карло для определения ядерной структуры, рассеяния и реакций. Его преимущество заключается в относительно мягком масштабировании с размером системы и общей платформе для обработки систем с несколькими и многими телами при нулевой и ненулевой температуре. Однако есть дополнительная трудность при работе с решеткой с нарушенной вращательной симметрией, и шаг решетки необходимо уменьшить, чтобы уменьшить систематические ошибки.
Мы также упоминаем несколько других недавних исследований. В одной недавней работе состояния 12-C рассматриваются в модели Скирма (Lau and Manton, 2014). Хотя расчеты хорошо согласуются с измеренным экспериментальным спектром, подробная связь с лежащими в основе ядерными силами еще не полностью реализована. В то время как недостатки оболочечной модели в описании кластерных структур были известны с ранних лет, объяснение ядерной кластеризации как возникающего коллективного явления вблизи открытых порогов дается в работе [5]. (Okolowicz et al., 2013), рассматривая ядро как открытую квантовую систему, взаимодействующую через близлежащие состояния континуума.


\subsection{ Последные экспериментальные результаты}
\subsubsection{ Экспериментальные наблюдаемые}
Экспериментальное изучение роли кластеризации в ядрах восходит к самым ранним наблюдениям $\alpha$-распада тяжелых ядер. В ранних моделях ядер многие предполагали, что $\alpha$-частица может играть важную роль, например, работа Хафстада и Теллера в 1938 г. (Hafstad and Teller, 1938) хорошо описывает возможные структуры ядер, таких как 8-Be , 12-C и 16-O, построенные из $\alpha$-частиц. В этой ранней работе также высказывались предположения о существовании молекулярных структур в легких ядрах, где нейтроны или даже нейтронные дыры могут обмениваться между ядрами $\alpha$-частиц. Эти основные идеи остаются движущими силами для большей части нынешней экспериментальной программы. Катализатором "современной" эры кластеризации ядер послужили идеи Моринаги в 1956 году, который предположил, что состояние Хойла с энергией 7,65 МэВ при 12-C, которое недавно было измерено экспериментально, могло быть линейным расположением 3$\alpha$-частиц (Morinaga, 1956). Идея о том, что линейные цепные структуры могут существовать в ядрах, до сих пор остается актуальной и остается нерешенной. Эксперимент был в значительной степени мотивирован желанием предоставить доказательства типов структур, предусмотренных Моринага и рассчитанных Бринком с использованием модели альфа-кластеров Блоха-Бринка (Brink, 2008; Brink and Boeker, 1967). Например, в случае 12-C модель $\alpha$-кластера обнаруживает две структуры. Первый - это равностороннее треугольное расположение, которое исторически ассоциировалось с основным состоянием, а второе - линейное расположение (или цепочка).

Возможность экспериментов выяснить кластерные структуры легких и тяжелых ядер определяется диапазоном экспериментальных наблюдаемых, которые могут быть извлечены. С упрощенной отправной точки, момент инерции вращающегося ядра дает представление о деформации, которая может быть, по крайней мере, показана как совместимая с кластерной структурой, даже если не является прямым доказательством. Если использовать 8 Be в качестве примера, то вращательная полоса основного состояния имеет состояния 0$^{+}$, 2$^{+}$ и 4$^{+}$ при 0, 3,06 и 11,35 МэВ. Отношение энергии 4$^{+}$ к 2$^{+}$ составляет 3,7, что очень близко к тому, что можно было бы ожидать для вращающегося ядра, 3,33. Момент инерции, извлекаемый из E$_{rot}$ = J (J +1) $\sim$ 2 / 2I, соизмерим с моментом, найденным в расчетах ab initio методом Монте-Карло (GFMC), которые четко раскрывают структуру кластера (Wiringa et al., 2000). Мы обсуждаем вычисления с использованием функции Грина Монте-Карло в подразделе VI.B. В качестве простого руководства, значение $\sim$ 2 / 2I, связанное с состоянием 2$^{+}$, составляет 0,51 МэВ, что даже при простом вычислении дает разделение двух $\alpha$-частиц на удвоенный радиус $\alpha$-частицы. Наблюдение за серией состояний, лежащих во вращательной последовательности, не является водонепроницаемым свидетельством кластеризации или деформации. Здесь измерения сил электромагнитных переходов обеспечивают проверку перекрытия структур в начальном и конечном состоянии и степени коллективности. Для случая 8 Be измерение силы перехода B (E2) из состояния 4$^{+}$ в состояние 2$^{+}$ обеспечивает согласованное описание как с вращательной картиной, так и с расчетами GFMC (Datar et al., 2013).

Однако, как отмечалось выше, к этой упрощенной интерпретации следует относиться с осторожностью. Во-первых, все состояния в 8-Be не связаны и, следовательно, встроены в континуум и, следовательно, будут иметь вклады континуума. Во-вторых, ширина состояний значительна и, соответственно, время жизни короткое, и поэтому понимание того, что означает коллективность в таких коротких временных масштабах, неясно. Наконец, многие вычисления используют приближения связанных состояний и, следовательно, не могут быть полностью точными. Существует интересное обсуждение значения вращательных полос, в которых резонансы встроены в континуум, с акцентом на 8 Be Гарридо и др. (Гарридо и др., 2013). Вывод состоит в том, что вращательные полосы, встроенные в континуум, все еще могут быть значимой концепцией, но что континуум влияет на такие свойства, как вероятности переходов, и, следовательно, здесь к континууму нужно обращаться осторожно. Это особенно важно для сравнения с методами ab initio.

Ширина состояния показывает значительное количество деталей, касающихся структуры и распада. Чем больше перекрытие исходной структуры с перегородкой распада, тем короче время жизни и больше ширина. В случае возбуждения 2$^{+}$ 8-Be ширина составляет 1,5 МэВ. На ширину распада также влияет барьер, через который распад должен проходить, но если убрать кулоновский и центробежный барьеры, то уменьшенную ширину можно сравнить с пределом Вигнера. Это значение, которое должна принимать уменьшенная ширина, если $\alpha$-частицы полностью предварительно сформированы. Для этого конкретного состояния было обнаружено, что экспериментальная ширина очень близка к пределу Вигнера, что снова указывает на существование кластерной структуры (Cerny, 1974; Overway et al., 1981). Еще одна сигнатура, недоступная для распада примерных состояний в 8-Be, - это измерение доминирующего канала распада. Состояния с сильными кластероподобными свойствами должны предпочтительно распадаться за счет излучения кластера, а не, например, распада протона или нейтрона. В реакциях это структурное сходство может быть описано с помощью спектроскопического фактора или асимптотического нормировочного коэффициента (ANC\nomenclature{ANC}{асимптотический нормировочный коэффициент}).

В следующих разделах мы исследуем многие недавние достижения в экспериментальном исследовании кластеризации ядер. Во многих случаях недавние работы основываются на значительных исторических трудах. Существует множество обзорных статей, описывающих развитие предмета, и мы отсылаем читателя к следующим источникам: (Beck, 2010, 2012, 2014; Freer and Fynbo, 2014; Freer and Merchant, 1997; Freer, 2007; von Oertzen et al. др., 2006).

\subsection{ Состояние исследований легких ядер}
\subsubsection{ Альфа-сопряженные системы. N -альфа-структуры и цепи}
\hspace{0.6cm}

Безусловно, наибольшее экспериментальное внимание было уделено изучению кластерной структуры $\alpha$-сопряженных ядер. Здесь проблема заключалась в том, чтобы сначала обеспечить более глубокое понимание природы кластерных структур и, в конечном итоге, определить, действительно ли цепные состояния существуют в легких ядрах или нет. Конечная цель состоит в том, чтобы определить экспериментальные характеристики, чтобы их можно было проверить с помощью ab initio или других микроскопических расчетов.
Быть
Как уже было описано, одним из лучших примеров сравнения между ab initio теорией и экспериментом является измерение гамма-распада состояния 4$^{+}$ в 8-Be в состояние 2$^{+}$ (Datar et al., 2013). Это был тур-де-сила, где наблюдалась ветвь гамма-распада $\sim$ 10$^{-7}$. В эксперименте использовалась мишень с газовой струей гелия, а состояние 4$^{+}$ резонансно заселялось пучком 4-He. Испускаемое гамма-излучение и последующее испускание двух $^{+}$-частиц от распада состояния 2$^{+}$ были обнаружены при тройном совпадении. Наблюдалось поперечное сечение 165 (54) нб, что переводилось в B (E2) 25 $\pm$ 8 e$^{2}$ фм$^{4}$. Это очень близко к последнему вычисленному в методе GFMC значению 26,0 $\pm$ 0,6 e$^{2}$ фм $^{4}$ (Datar et al., 2013). Эти последние расчеты, как известно, показали, что основное состояние 8-Be сильно кластеризовано, и со значительной точностью предсказали спектр энергии возбуждения (Wiringa et al., 2000). Учитывая, что B (E2) чувствителен как к перекрытию распределения зарядов, так и к коллективному поведению, такой результат можно рассматривать как свидетельство как кластерного, так и коллективного поведения. Однако в этом случае возникает довольно интересная загадка. 1 Ширина состояний 2$^{+}$ и 4$^{+}$ велика (1,5 и 3,5 МэВ соответственно). Из принципа неопределенности это соответствует времени жизни порядка 10$^{-22}$ секунды. Это время пролета нуклона с энергией Ферми через ядро. Как могут развиваться коллективные процессы и возникать ротационное поведение при очевидном несоответствии во временных масштабах и что означает ротация в таких системах (Fossez et al., 2016). Когда дело доходит до точного описания свойств таких состояний, встроенных в континуум, необходимо полностью учитывать влияние континуума на свойства перехода (Garrido et al., 2013), и очень важно, чтобы методы ab initio\nomenclature{Ab initio}{обоснование какого-либо явления из естественных законов природы без привлечения дополнительных эмпирических предположений или специальных моделей} были разработаны для такие несвязанные системы.

Состояние Хойла в 12-C - одно из наиболее известных состояний в ядрах, учитывая его довольно важную роль в синтезе углерода посредством тройного $\alpha$-процесса. Недавний обзор этого состояния (Freer and Fynbo, 2014) дает исчерпывающее описание его роли в синтезе и его экспериментальных свойств. Достаточно сказать, что с экспериментальной точки зрения эти свойства хорошо изучены. С другой стороны, его структура менее изучена.

Тот факт, что в расчетах модели оболочки без ядра не удается воспроизвести энергию состояния Хойла (Navr\`{a}til et al., 2007, 2000b), без использования значительно расширенного базиса гармонических осцилляторов, уже указывает на то, что структура лежит за пределами того, что легко описывается модель оболочки. Первый ab initio расчет состояния Хойла был проведен всего несколько лет назад в [4]. (Эпельбаум и др., 2011). Эти последние вычисления смогли явно зафиксировать $\alpha$-кластеризацию, которая появляется в этом состоянии. Расчеты AMD, рис. 1, показывают, что состояние Хойла представляет собой расширенную трех $\alpha$-систему и что связанные с ней возбужденные состояния 2$^{+}$ и 4$^{+}$ не являются жесткими, вращательными, возбуждениями и что рыхлая совокупность $\alpha$-частиц, -газ, может быть лучшее описание. К аналогичному выводу пришли расчеты фермионной молекулярной динамики (FMD\nomenclature{FMD}{фермионная молекулярная динамика}) для тех же состояний (Neff, Feldmeier, 2014). Здесь было высказано предположение, что резонансы 2 + и 4 + могут рассматриваться как члены вращательной полосы, построенной на основном состоянии 8-Be, где третья $\alpha$-частица вращается вокруг ядра 8-Be с относительным орбитальным угловым моментом 2 или 4 соответственно. Происхождение ядерных кластеров, имеющих отношение к образованию состояния Хойла, также обсуждается Okolowicz et al. (Okolowicz et al., 2013).

Баркер и Трейси (Barker and Treacy, 1962) заметили, что для воспроизведения ширины состояния Хойла необходимо использовать необычно большой радиус: с радиусом 1,6 фм A 1/3, шириной 9,3 эВ соответствует безразмерной приведенной ширине $\theta_{2}$ = $\gamma\lambda 2 M_{red}R_{2}/2\sim 2$ , равной 1,5. Следовательно, ширина состояния Хойла очень велика; это можно понять только при наличии большой степени $\alpha$-кластеризации. Наличие такой кластерной структуры увеличивает сечение $\alpha$-захвата. Но его наличие в окне Гамова приводит к увеличению общего сечения захвата в 10$^{8}$ раз. Без точного местоположения этого состояния количество углерода-12 было бы значительно уменьшено, и, таким образом, оно тесно связано с существованием органической жизни. Довольно глубокий вопрос заключается в том, является ли это счастливой случайностью, или есть какая-то причина, по которой состояния с сильно развитой кластерной структурой должны существовать вблизи соответствующих пороговых значений распада (Epelbaum et al., 2013b, a; Freer and Fynbo, 2014; Okolowicz et al., 2013).

Помимо того факта, что состояние Хойла имеет 3$\alpha$-кластерную структуру, природа этой структуры еще предстоит выяснить. Расчеты AMD на рисунке 1 показывают преобладание $^{8}$Be + $\alpha$ конфигураций в рыхлой сборке, так что 2$^{+}$ и 4$^{+}$ возбуждения не обладают четким вращательным поведением. Расчеты фермионной молекулярной динамики (FMD) состояния Хойла приводят к аналогичным выводам (Chernykh et al., 2007). Расширением этих идей является то, что состояние может быть описано газом / конденсатом $\alpha$-частиц (Funaki et al., 2009). В принципе, можно получить представление о структуре через свойства распада состояния. В этом случае открыты две моды распада; последовательный и прямой. В последнем случае система не распадается через основное состояние 8-Be. Верхний предел непоследовательного $\alpha$-распада в 4\% был впервые определен в 1994 г. (Freer et al., 1994). Впоследствии измерение реакции $^{40}$Ca + $^{12}$C при 25 МэВ / нуклон показало, что степень разветвления на самом деле была выше и составляла 7,5 $\pm$ 4\%. Это было оспорено дальнейшими измерениями, в которых были предложены верхние пределы всего 5 $\cdot$ 10 $^{-3}$ (95\% CL) (Kirsebom et al., 2012; Manfredi et al., 2012) и 9 (2)$\cdot$ 10$^{-3}$ (Рана и др., 2013). Он был улучшен до 0,2\% (Itoh et al., 2014). Эти измерения теперь достигли чувствительности, при которой эффекты фазового пространства перестают быть доминирующим фактором, и можно исследовать структуру с пределами 0,047\% (Smith et al., 2017) и 0,043\% (Dell'Aquila et al., 2017) по сравнению с прогнозируемым пределом фазового пространства 0,06\% (Smith et al., 2017).

Второй подход заключается в исследовании распределения заряда посредством неупругого рассеяния электронов (Хорикава и др., 1971; Накада и др., 1971; Sick and Mccarthy, 1970; Strehl and Schucan, 1968). В таких измерениях определяется переходный форм-фактор, который исследует перекрытие основного состояния с состоянием Хойла. Для интерпретации таких измерений требуется модель, которая может описывать как основное, так и возбужденное состояния. И описание конденсата (Funaki et al., 2006a), и описание FMD (Chernykh et al., 2007) указывают на то, что состояние Хойла связано с радиусом, большим, чем у основного состояния, в 1,35-1,60 раза (в зависимости от модель, использованная для анализа данных), что соответствует увеличению объема в 2,5-4 раза. На рисунке 2 показано рассчитанное распределение неупругого рассеяния электронов для модели конденсата (Funaki et al., 2006a).

Третий подход к выводу структуры состояния Хойла - поиск коллективных возбуж-дений, в частности возбуждения 2$^{+}$. Измерения неупругого рассеяния (Freer et al., 2009; Itoh et al., 2011; Zimmerman et al., 2011) были первыми, кто подтвердил наличие такого возбуждения. Общий анализ данных о рассеянии протонов и $\alpha$-частиц в пользу 2$^{+}$-резонанса дан в [5]. (Freer et al., 2012a), и обсуждение влияния этих измерений дано в работе. (Финбо и Фрир, 2011). Форма линии 2$^{+}$, которая была обнаружена в измерениях неупругого рассеяния 12-C ($\alpha$, $\alpha_{0}$) и 12-C (p, p$_{0}$) (Freer et al., 2012a), определила свойства как E$_{x}$ = 9,75 ( 0,15) МэВ при ширине 750 (150) кэВ. Существование 2$^{+}$ -резонанса было подтверждено измерением реакции 12-C ($\gamma$, 3$\alpha$) на установке HI$\gamma$S (Zimmerman et al., 2013). Функция возбуждения для этих измерений показана на рисунке 3 и дает резонансные параметры E$_{x}$= 10,13 (6) МэВ и $\Gamma$ = 2,1 (3) МэВ (Zimmerman, 2013).

[рис.2]

РИС.2  Рассчитанный неупругий форм-фактор для неупругого рассеяния электронов из основного состояния 0$^{+1}$ в возбужденное состояние 0$^{+2}$ (Funaki et al., 2006a) для подхода BEC (красный) по сравнению с экспериментальным данные из исх. (Horikawa et al., 1971; Nakada et al., 1971; Sick and Mccarthy, 1970; Strehl and Schucan, 1968).

Эти измерения теперь расширены до более высоких энергий и продолжают ожидаемую тенденцию для возбуждения 2$^{+}$. Если состояние имеет вращательное поведение, тогда также должно быть состояние 4+, близкое к 14 МэВ. Существуют предварительные доказательства такого состояния при 13,3 МэВ и ширине 1,7 МэВ (Freer et al., 2011; Jyv$\ddot{a}$skyl$\ddot{a}$, 2013; Ogloblin et al., 2014). Существование этого последнего состояния еще предстоит окончательно подтвердить. Похоже, что он сильно распадается на основное состояние 8-Be, в отличие от возбужденного состояния 2$^{+}$, что может дать представление о том, как строится угловой момент, то есть через вращение $\alpha$-частицы вокруг 8-Be (0$^{+}$) ядро. Хотя был достигнут большой прогресс в понимании структуры 12-C, измерения обычно являются сложными и часто далеко не однозначными. Таким образом, потребность в подробной спектроскопии сохраняется. Здесь подход Орхусской группы (Kirsebom et al., 2014) к измерению электромагнитных свойств указывает путь для этих будущих исследований. Реакция захвата p + $^{11}$B используется для резонансного заселения состояний при 12-C, и их распад после испускания ненаблюдаемого гамма-распада регистрируется через следующий канал заряженных частиц.

Состояние Хойла, хотя и расширенное, не согласуется с линейной структурой цепочки, которая требует, чтобы состояние 2+ лежало на ~ 1 МэВ ниже, чем наблюдаемое экспериментально. Теоретические указания (Kanada-En'yo, 2007) предполагают, что состояние 10,3 МэВ, 0 + 3 является наилучшей возможностью. Это состояние имеет ширину 3 МэВ, и можно ожидать, что состояние 2+, соответствующее линейной цепочечной структуре, близко к 11,5 МэВ, и будет иметь очень большую ширину. Пока такое состояние еще предстоит наблюдать. Недавно Ито и др. Экспериментально сообщили о возможности двух состояний 0+ около 10 МэВ. (Itoh et al., 2011) и поддерживается расширенным расчетом Tohsaki-Horiuchi-Schuck-R$\ddot{o}$pke (THSR) (Funaki, 2015; Funaki et al., 2015).

[рис.3]

РИС.3  (а) Измеренные сечения E$_{1}$ и E$_{2}$ реакции 12-C ($\gamma$, $\alpha_{0}$) 8-Be. (b) Измеренный относительный фазовый угол E$_{1}$-E$_{2}$ ($\varphi$ 12) вместе с фазовым углом, рассчитанный по двухрезонансной модели (Zimmerman et al., 2013).

На рис. 4 показана компиляция теоретических спектров и переходов для состояний 0$^{+}$ и 2$^{+}$ в сравнении с экспериментальными данными. Хотя существует множество не- и полумикроскопических расчетов 3$\alpha$, мы показываем только микроскопические расчеты с полностью антисимметризованными волновыми функциями и нуклон-нуклонными взаимодействиями. Трудно напрямую сравнивать качество воспроизведения микроскопических расчетов с немикроскопическими расчетами, где взаимодействия (или гамильтониан) обычно феноменологически корректируются, чтобы соответствовать энергетическим спектрам 12-C. Следует также отметить, что мы не должны обсуждать ab initio вычисления. полученные из реальных ядерных сил на одной основе с расчетами с использованием феноменологических эффективных ядерных взаимодействий.  В 3$\alpha$RGM (Kamimura, 1981), расширенном THSR (Funaki, 2015; Funaki et al., 2015), 3$\alpha$GCM (Descouvemont, Baye, 1987; Suhara, Kanada-En'yo, 2015; Uegaki et al., 1979), и 3$\alpha$+ p 3/2 (Suhara, Kanada-En'yo, 2015) используются феноменологические эффективные ядерные взаимодействия волковских сил (Волков, 1965). Параметры взаимодействия сил Волкова настроены так, чтобы воспроизвести $\alpha$-$\alpha$-рассеяние, хотя между этими расчетами есть небольшие различия в параметрах. Результаты AMD (Kanada-En'yo, 1998a, 2007) получены с использованием силы MV1 (Ando et al., 1980), которая представляет собой феноменологическое эффективное ядерное взаимодействие, модифицированное на основе силы Волкова для описания свойств насыщения, тогда как Результаты FMD + 3$\alpha$ (Черных и др., 2007) получены на основе реалистичного аргоннского потенциала V18 с феноменологической настройкой. Для расчетов NCSM (Navratil et al., 2007) и теории эффективного поля ядерной решетки (NLEFT)\nomenclature{NLEFT}{Nuclear Lattice Effective Field Theory - теория эффективного поля ядерной решетки} (Epelbaum et al., 2012) показаны результаты, полученные с реалистичными силами NN и NNN, полученными из киральной эффективной теории. В расчетах симплектической модели без ядра (NCSpM) (Dreyfuss et al., 2013) используется упрощенный эффективный гамильтониан.

В целом расчеты 3$\alpha$ хорошо описывают энергетические спектры состояний кластера выше порога 3$\alpha$ и формфакторы рассеяния электронов для состояния 0${+1}$ и перехода 0$^{+2}$ $\rightarrow$ 0$^{+1}$, но их недостаточно для описания некоторых свойств низколежащие состояния, такие как расстояние между уровнями 0$^{+1}$ -2$^{+1}$, сила перехода $E_{2}$ для 2$^{+1}$ $\rightarrow$ 0${+1}$ и 0$^{+2}$ $\rightarrow$ 2 1. Гибридные расчеты моделей 3$\alpha$ + p 3/2 и FMD + 3$\alpha$, а также AMD могут разумно описать свойства основной полосы и возбужденные спектры для состояний кластера. Расчет NCSM не может описать возбужденные состояния кластера выше порога, поскольку эти состояния находятся за пределами модельного пространства, тогда как NCSpM, который содержит более высокие конфигурации оболочки для возбуждений кластера, и вычисления NLEFT описывают структуры кластера в возбужденных состояниях выше порога. например, расстояние между уровнями 0$^{+1}$ - 2$^{+1}$, сила перехода E$_{1}$ для 2$^{+1}$ $\rightarrow$ 0$^{+1}$ и 0$^{+2}$ $\rightarrow$ 2$^{+1}$. Гибридные расчеты моделей 3$\alpha$ + p 3/2 и FMD + 3$\alpha$, а также AMD могут разумно описать свойства основной полосы и возбужденные спектры для состояний кластера. Расчет NCSM не может описать возбужденные состояния кластера выше порога, поскольку эти состояния находятся за пределами модельного пространства, тогда как NCSpM, который содержит более высокие конфигурации оболочки для возбуждений кластера, и расчеты NLEFT описывают кластерные структуры в возбужденных состояниях выше порога. Расчеты ab initio (NCSpM и NLEFT) имеют тенденцию сильно недооценивать размер основного состояния, а также дают малые значения размера и матричного элемента E0 для состояния Хойла. Ширины $\alpha$-распада рассчитаны в 3$\alpha$RGM (Kamimura, 1981) и 3$\alpha$GCM (D) в [5]. (Descouvemont and Baye, 1987) путем решения $^{8}$Be + $\alpha$-рассеяния и оценены в расширенном THSR (Funaki, 2015; Funaki et al., 2015), 3$\alpha$GCM (U) в Ref. (Uegaki et al., 1979) и AMD (Kanada-En'yo, 2007) в рамках приближений связанных состояний с использованием уменьшенных амплитуд ширины. Имеющиеся данные для ширин $\alpha$-распадов количественно или качественно воспроизводятся теоретическими расчетами.

Хотя в понимании структуры 12-C был достигнут значительный прогресс, очевидно, что существуют как необходимость, так и возможности для измерений, чтобы более точно ограничить свойства состояний, представленных на рисунке 1. В частности, это требует измерения скорости электромагнитных переходов там, где это возможно.


\section{Методика эксперимента}
\hspace{0.6cm}
Представленные в настоящей диссертационной работе физические
результаты основываются на анализе экспериментальных данных по
кислород-протонным взаимодействиям при 3.25 ГэВ/с на нуклон, которые
были получены с помощью 1 м водородной пузырьковой камеры (ВПК\nomenclature{ВПК}{Водородная пузырьковая камера}) в
рамках сотрудничества с Лабораторией высоких энергий (ЛВЭ)
Объединенного института ядерных исследований (ОИЯИ). ВПК ЛВЭ ОИЯИ
экспонировалась в пучке релятивистских ядер кислорода на Дубненском
синхрофазотроне.

Методика пузырьковых камер при изучении множественной генерации
частиц и процессов фрагментации ядер в адрон- и ядро-ядерных соударениях при высоких энергиях имеет ряд достоинств, а именно, возможность разделения частиц и фрагментов (ядра-снаряда) по зарядам, что невозможно в экспериментах с фотоэмульсиями без магнитного поля, с хорошей точностью определения их импульса в условиях близких к 4$\pi$-геометрии.

Основными недостатками пузырьковых камер являются, во-первых, их
низкое быстродействие и длительный процесс обработки стереоснимков; во-вторых, невозможность регистрации медленных фрагментов ядра-мишени
из-за малости длины их пробега в рабочем объеме камеры (L $\le$0.2-0.3 см.).

Известными достоинствами ВПК является высокая точность импульсных и угловых измерений, наблюдаемость акта взаимодействия, хорошее пространственное разрешение, соответствующее 4$\pi$-геометрии, надежная идентификация фрагментов по ионизации и импульсам, а также, в силу уникальности эксперимента с налетающим ядром, возможность
регистрации мало энергичных в системе покоя ядра кислорода заряженных
частиц, включая испарительные. Эксперименты в водородной камере дают
дополнительное преимущество поскольку, в отличие от других
экспериментов, рабочая жидкость (мишень) является однородной по
химическому составу и не вносит неопределенностей, связанных, как,
например, в фотоэмульсии со статистическим разделением вкладов ядер-
мишеней на компоненты с определенным массовым числом.
\subsection{Диссоциация релятивистских ядер}
\hspace{0.6cm}
Эмульсионная камера собирается как стопка слоев с толщиной около 550 мкм и размерами 10х20 см$^{2}$ (рис. 4). Факторами получения значительной статистики событий оказываются толщина, которая достигает 80 г/см$^{2}$ вдоль длинной стороны, и полная эффективность детектирования заряженных частиц. В ядерной эмульсии содержатся в близких концентрациях как тяжелые ядра Ag и Br, так и ядра H. По плотности водорода ядерная эмульсия близка к жидководородной мишени. Эта особенность позволяет сравнивать в одинаковых условиях развалы ядер-снарядов как в результате дифракционной или электромагнитной диссоциации на тяжелом ядре-мишени, так и в результате столкновений с протонами. 


Рис. 4. Фотография слоя ядерной эмульсии на стеклянной подложке и микроскопа с установленной фотокамерой.

Фрагменты релятивистского ядра сосредоточены в конусе, ограниченном углом
\begin{equation}
    \sin\theta_{fr}=\frac{p_{fr}}{p_{0}}
    \label{1}
\end{equation}
где $p_{fr}$ = 0.2 ГэВ/c – величина, характеризующая Ферми-импульс нуклонов, а $p_{0}$ – импульс на нуклон ядра-снаряда. Если пучок направляется параллельно плоскости слоев, следы всех релятивистских фрагментов остаются достаточно долго в одном слое для 3-мерной реконструкции. Распределение событий по каналам взаимодействий с различным составом заряженных фрагментов (или зарядовая топология) является центральной характеристикой фрагментации релятивистских ядер. Результаты по зарядовой топологии когерентной диссоциации для релятивистских ядер 16-O, 22-Ne, 24-Mg, 28-Si и 32-S суммированы в работе [46].

В ядерной эмульсии угловое разрешение для следов релятивистских фрагментов составляет величину порядка 10$^{-5}$ рад. Измерения полярных углов вылета фрагментов $\theta$ оказываются недостаточными для сравнения данных при различных значениях начальной энергии ядер. Более универсальным является сравнение по величинам поперечных импульсов $P_{T}$ фрагментов с массовым числом $A_{fr}$ согласно приближению
\begin{equation}
    P_{T}=A_{fr}p_{0}\sin\theta
    \label{2}
\end{equation}
что соответствует сохранению фрагментами скорости первичного ядра (или импульса на нуклон P$_{0}$ ). Очевидно, что наибольшее значение имеет разрешение по углу $\theta$, распределение по которому "прижато" к нулю. Для ядер с $\alpha$-кластерной основой является оправданным предположение о соответствии релятивистского фрагмента с зарядом $Z_{fr}$= 2 изотопам 4-He.

Для нейтронодефицитных ядер требуется разделение изотопов 3-He и 4-He. При фрагментации ядер состава эмульсии могут наблюдаться сильноионизирующие фрагменты мишени (рис. 2), включая $\alpha$-частицы, протоны с энергией ниже 26 МэВ и легкие ядра отдачи – n$_{b}$ (b-частицы), а также нерелятивистские протоны с энергией свыше 26 МэВ – n$_{g}$ (g-частицы). Кроме того, реакции характеризуются множественностью мезонов, рожденных вне конуса фрагментации – n$_{s}$ (s-частицы). По этим параметрам можно сделать вывод о характере взаимодействия. В событиях когерентной диссоциации отсутствуют как фрагменты ядер мишени (n$_{b}$ = 0, n$_{g}$ = 0), так и заряженные мезоны (n$_{s}$ = 0). События такого типа из-за отсутствия следов сильноионизирующих частиц nh (nh = nb + ng ) получили неформальное наименование "белых" звезд. "Белые" звезды возникают при ядерном дифракционном и электромагнитном взаимодействии на тяжелых ядрах мишени. Их доля от общего числа неупругих событий составляет несколько процентов. Название "белые" звезды удачно отражает также "срыв" ионизации при переходе от следа первичного ядра к узкому конусу вторичных следов вплоть до Z$_{pr}$ раз. Эта особенность составляет основную трудность для электронных методов, поскольку чем больше в событии степень диссоциации, тем труднее его зарегистрировать. В ядерной эмульсии ситуация совершенно обратная. 	
\subsection{Сечения $^{16}$Ор-взаимодействий}
\hspace{0.6cm}
В результате просмотра в рабочей зоне камеры, составляющей 44 см,
найдено 17448 $^{16}$Ор-взаимодействий на 39847 пучковых частиц. Когда плотности жидкого водорода $\rho$=0.0584 $\pm$0.0001 г/см$^{3}$, полное наблюдаемое сечение составило $\sigma_{tot}$=375 $\pm$ 9 мбн. Оно требует уточнения и поправок из-за незарегистрированных событий. В первую очередь–это упругие события,
характеризующиеся малыми углами отклонения вторичного трека. Для
оценки числа потерянных двухлучевых событий было проанализировано
распределение по переданному протону-мишени 4-импульсу. Экстраполируя
распределение $d\sigma/dt$ в область $|t|$=0 и учитывая, что доля упругих рассеяний составляет $\sim$1/2 общего числа двухлучевых событий, можно получить нижнее значение сечения потерь – 20 $\pm$2 мбн [106; С.648-655.] и, соответственно, полное сечение $^{16}$Ор-взаимодействий $\sigma_{tot}$=395$\pm$10 мбн. Когда
учтены всех поправок и корректировок на потери событий при просмотре, неизмеримостью коротких, летящих в бок протонов отдачи и доли упругих
рассеяний среди двухлучевых событий неупругое сечение $^{16}$Ор-
взаимодействий оказалось равным $\sigma_{in}$=334 $\pm$6 мбн [105; С.2-9.].Интересно сравнить полученное значение с мировыми данными, в частности, с популярной аппроксимацией мировых данных по неупругим сечениям адрон-ядерных взаимодействий [107; С.854-858.]:
\begin{equation}
    \sigma_{in}(A,T) = 47,29 A^{1/3}(A^{1/3} + 0,039 X - 0,0009X^{2})
\end{equation}
где $X=\sigma_{0}(T)-33$ - параметр разложения $\sigma_{in}(A,T)$ в ряд по степеням Х; А–
массовое число ядра-мишени, Т–энергия налетающей частицы, $\sigma_{0}$ (Т)–близкая к полному сечению нуклон-нуклонного или
пион-нуклонного взаимодействия, подлежащая определению из близких по энергии первичной частицы экспериментов. В соответствии с [108; С.427-435.] мы взяли
$\sigma_{0}$(Т)=42.22 и получили неупругое сечение $^{16}$Ор-взаимодействий равным $\sigma_{in}$ =332$\pm$ мбн, прекрасно согласующееся с нашими данными.



\section{Образование многонуклонных систем и ядер в $^{16}$Ор-взаимодействиях}
\subsection{Исследование процессов фрагментации релятивистских ядер при периферических соударениях}
\hspace{0.6cm}
Исследование процессов фрагментации релятивистских ядер при периферических соударениях с нуклонами и ядрами позволяет получить важную информацию о возможной кластерной структуре фрагментирующего ядра и влиянии ее на состав и выход конечных продуктов реакции. В последние годы сотрудничеством "Беккерель" [см. напр., 1–6] проводятся исследования кластерных структур легких релятивистских ядер в их периферических соударениях с ядрами фотоэмульсии и получены интересные результаты. В частности, показано [1-6], что в периферических соударениях легких релятивистских ядер с ядрами фотоэмульсии (когда ядро-мишень не фрагментирует) с большей вероятностью проявляются не только $\alpha$-кластерные, но и $\alpha$+2Н, $\alpha$+3Н, $\alpha$+3-Не и тому подобные кластерные структуры. Состав кластерной структуры легких ядер определяется их массовым числом и четностью или нечетностью количества протонов и нейтронов в них [1-6]. При исследовании развала ядер кислорода во взаимодействиях с протонами при 3.25 А ГэВ/с на многозарядные фрагменты с сохранением в них заряда исходного ядра было обнаружено, что из экспериментально наблюдаемых трех топологий (224), (2222) и (26) (цифры в скобках означают заряд многозарядного фрагмента, а их число - количество указанных фрагментов) только в последних
двух наблюдаются события, в которых сохраняется также исходное число нуклонов ядра кислорода. Среди этих топологий максимальное сечение выхода имеет топология (26) – 10.14 мбн, а минимальное – топология (224) – 0.93 $\pm$ 0.18 мбн [7]. В работе [8] были исследованы характеристики трехлучевых событий топологии (26) с сохранением в них протона отдачи для поиска и анализа дифракционного развала ядра кислорода на $\alpha$-частицу и ядро углерод-12. На основании анализа формы спектра по разности азимутальных углов $\alpha$-частицы и протона отдачи было получено указание на то, что развал ядра кислорода на $\alpha$-частицу и ядро углерод-12 может происходить двумя способами: 1) прямой развал в результате возбуждения ядра кислорода как целое – дифракционный механизм; 2) развал ядра кислорода за счет квазиупругого выбивания протоном мишенью одного из четырех его $\alpha$-кластеров. Настоящая работа является продолжением [8] и посвящена детальному анализу топологии [26]. Экспериментальный материал был получен с помощью 1-м водородной пузырьковой камеры ЛВЭ ОИЯИ\nomenclature{ЛВЭ}{Лаборатории высоких энергий}, облученной пучком ядер 16О с импульсом 3.25 А ГэВ/с, на Дубненском  синхрофазотроне ОИЯИ\nomenclature{ОИЯИ}{ Объединенный Институт Ядерных Исследований}. Данные, анализируемые в этой работе, получены из 8712 полностью измеренных $^{16}$Ор-событий.

Для идентификации фрагментов по массе были введены следующие интервалы импульса в лабораторной системе координат: однозарядные фрагменты с 1.75 $\ge$ p < 4.75 ГэВ/с считались протонами, с p =4.75-7.75 ГэВ/с относились к 2-Н и с p > 7.75 ГэВ/с к ядрам 3-Н. Двухзарядные фрагменты с p < 10.75 ГэВ/с относились к 3-Не, а с p $\ge$ 10.75 ГэВ/с к 4-Не. Шестизарядные фрагменты с p < 34.1 ГэВ/с считались ядрами 10-С, с 34.1$\le$p < 37.35 ГэВ/с – ядрами 11-С, с 37.35 $\le$ p$\le$ 40.6 ГэВ/с ядрами 12-С, а с p > 40.6 ГэВ/с относились к ядрам 13-С. Такое разделение фрагментов по массе и заряду позволяет определить множественность связанных нейтронов в многозарядных фрагментах в индивидуальных актах соударений. В связи с тем, что число связанных протонов в многозарядных фрагментах изучаемой нами топологии (26) равно 8 и используя законов сохранения электрического и барионного зарядов также можно определить множественность несвязанных нейтронов. Рассмотрим распределения событий по числу заряженных частиц (включая много-зарядных фрагментов) из топологии (26) в зависимости от наличия или отсутствия в них протона отдачи. В табл. 1 приведено число событий топологии (26) в зависимости от количества заряженных частиц n$_{z}$ в них с (без) сохранением протона отдачи n$_{pr}$.

\begin{table}
\begin{tabularx}{\textwidth}{|*{5}{Y|}}
\hline
\multirow{2}{*}{\textbf{n}$_{pr}$} 
  &\multicolumn{3}{c|}{\textbf{n}$_{z}$}&\textbf{Всего}\\
\cline{2-4}
             &3       &5     &7     &\\
\hline
\textbf{0}           &35       &8     &-    &43 \\
\hline
\textbf{1  }         &173      &18    &2    &193\\
\hline
\textbf{Все }        &208      &26    &2    &236\\
\hline
\end{tabularx}
\caption{Распределение событий по числу заряженных частиц n$_{z}$ из топологии (26) в зависимости от наличия или отсутствия протона отдачи n$_{pr}$.}
\label{tab:table1}
\end{table}

Как видно из табл. \ref{tab:table1}, основную часть (88\%) топологии (26) составляют события с тремя заряженными частицами (трехлучевые) в конечном состоянии и в них события с протоном отдачи составляют около 90\%. Также видно, что и в целом в топологии (26) доля событий с протоном отдачи составляет около 82\%, а без протонов отдачи – 18\%. Из табл. \ref{tab:table1} также можно заметить, что не наблюдается ни одного семилучевого события без протона отдачи, т.е. события с образованием 5 заряженных пионов из которых три – положительные и два – отрицательные. По-видимому, из-за малости первичной энергии образование пяти заряженных пионов в одном событии является маловероятным. В семилучевых событиях с протоном отдачи не наблюдается образования протона-фрагмента. Анализ состава однозарядных фрагментов в топологии (26) показал, что нет ни одного события с образованием дейтрона или трития в конечном состоянии. Так как в топологии (26) образование однозарядного фрагмента протона, дейтрона или трития связано с процессами неупругой перезарядки нейтрона снаряда (n $\rightarrow$ p + $\pi^{-}$) или передачей заряда протона-мишени одному из нейтронов ядра кислорода (np $\rightarrow$ pn) или многонуклонной системе – ядру-остатку, то это подтверждает тот факт, что заряд протона-мишени практически не участвует в формировании многонуклонных фрагментов [4]. Рассмотрим средние множественности заряженных
пионов, протонов отдачи, и протонов-фрагментов в событиях из топологии (26) (см. табл. \ref{tab:table2}).


\begin{table}
    \centering
    \begin{tabular}{|c|c|c|c|}
    \hline
       <n($\pi^{-}$)>  &<n($\pi^{+}$)> &n$_{pr}$&n$_{p}$ \\
       \hline
         0.13$\pm$0.02&	0.15$\pm$0.02&	0.82$\pm$0.05&	0.16$\pm$0.02\\
         \hline
    \end{tabular}
    \caption{Средние множественности заряженных пионов (<n($\pi^{-}$)>, <n($\pi^{+}$)>), протонов отдачи <n$_{pr}$> и протонов-фрагментов (<n$_{p}$>) в топологии (26).}
    \label{tab:table2}
\end{table}

Из табл. \ref{tab:table2} видно, что в подавляющей части событий топологии (26), как отмечалось выше, сохраняется протон отдачи. Средние множественности заряженных пионов и протонов-фрагментов в пределах статистических погрешностей совпадают друг с другом. Из-за закона сохранения электрического заряда в трехлучевых событиях с протоном отдачи не происходит образования протонов-фрагментов. В эксперименте образования протонов-фрагментов наблюдается в событиях с тремя заряженными частицами без протонов отдачи, и с пятью заряженными частицами в конечном состоянии. Как было указано выше экспериментально наблюдаемые протоны-фрагменты в топологии (26) могут быть образованы за счет неупругой перезарядки нейтрона снаряда       n$\rightarrow$ p + $\pi^{-}$ или в результате передачи заряда протона-мишени одному из нейтронов ядра кислорода np$\rightarrow$ pn. Анализ показал, что вклады первого и второго процессов в образовании протона-фрагмента в пределах статистических погрешностей совпадают друг с другом и соответственно равны 56 $\pm$ 12\% и      44 $\pm$ 12\%. Теперь перейдем к анализу изотопных составов двух- и шестизарядных фрагментов в топологии (26) (табл. \ref{tab:table3}). Как видно из табл. \ref{tab:table3} среди двухзарядных фрагментов наибольшую вероятность выхода имеют $\alpha$-частицы, а среди шестизарядных – ядро 12-С. 

\begin{table}[h]
    \centering
    \begin{tabular}{|c|c|c|c|c|c|}
    \hline
        \multicolumn{2}{|c|}{Изотопы ядра гелия}&\multicolumn{4}{c|}{Изотопы ядра углерода}\\
        \hline
        3Не&	4Не&	10С&	11С&	12С&	13C\\
        \hline
        0.16$\pm$0.03&	0.84$\pm$0.06&	0.15$\pm$0.03&	0.20$\pm$0.04&	0.61$\pm$0.05&	0.04$\pm$0.01\\ \hline

    \end{tabular}
    \caption{Изотопный состав двух- и шести зарядных фрагментов в топологии (26)}
    \label{tab:table3}
\end{table}
\subsection{Экспериментальные данные по изучению фрагментации ядер
кислорода}
\hspace{0.6cm}
Исследования процессов фрагментации ядер кислорода в условиях 4$\pi$-
геометрии во взаимодействиях с протонами при 3.25А ГэВ/с в 1 м
пузырьковой камере были начаты еще в 1989 г. и накоплен достаточно
большой экспериментальный материал.
В работе [49; С.413-417.] были исследованы множественности
различных типов заряженных частиц и фрагментов в
$^{16}$Ор-соударениях при 3.25 А ГэВ/с и были получены следующие, важные результаты.
\begin{enumerate}
    \item Показано, что основной вклад в множественность дают процессы фрагментации начального ядра кислорода, вклад же рожденных пионов оказываются маленьким. Распределение по множественности отрицательных частиц оказывается уже распределения Пуассона. Распределение по множественности положительных заряженных частиц имеет двугорбый характер.
    \item Проведено детальное исследование корреляций между множественностями частиц различного вида. Установлено, что на характер корреляций множественности сильное влияние оказывают законы сохранения электрического и барионных зарядов.
    \item Определены топологии и сечения (вероятности) осуществления различных каналов фрагментации ядра кислорода. Показано, что в процессах мультифрагментного развала кислорода особую роль играют каналы с образованием ядер гелия.
    \item Проведено предсказаниями детальное сопоставление каскадно-фрагментационной полученных данных испарительной с модели (КФИМ). Показано, что в целом процесс образования фрагментов в $^{16}$Ор-соударениях качественно объясняется с помощью использованных в модели механизмов, включающих: 
    \begin{enumerate}
        \item взаимодействие отдельных нуклонов ядра-снаряда с протоном, с учетом последующего перерассеяния продуктов этих взаимодействий
        \item распада на фрагменты образовавшегося возбужденного термализованного ядра
    \end{enumerate}
    Имеющиеся расхождения подчеркивают необходимость учета в КФИМ\nomenclature{КФИМ}{Каскадно-фрагментационная испарительная модель} кластерной структуры легких ядер, приводящей, в частности, к увеличению их вероятности распада на несколько $\alpha$-частиц.
\end{enumerate}
В исследовании структурной функции протонов для $^{16}$Ор-взаимодействий при 3.25 А ГэВ/с было показано, что механизм образования быстрых протонов с р> 0.25 ГэВ/с, особенно летящих вперед, имеет универсальный характер, т.е. не зависит от первичной
энергии, сорта налетающей частицы и типа легкого фрагментирующего ядра, а также от степени возбуждения фрагментирующих ядер [54; С.736-742. 55; С.216-222.].


\subsection{Просмотр стереоснимков с ВПК, отбор событий и измерения}
\hspace{0.6cm}
Стереографии просматривались на проекционных столах cувеличением
порядка 10 не менее трех раз. Применили метод двукратного просмотра
треков “по следу” и контрольного сравнения результатов первых двух
просмотров. Это позваляеть выявить все особенности регистрируемых актов взаимодействия кислорода в ВПК. Эффективность двойного просмотра для звезд с числом вторичных заряженных частиц n ch $\ge$ 3 оказалась близкой к 100\%, а для двухлучевых составляла порядка 90\%. В отборе первичных ионов кислорода ( 16-О) использовалось равенство суммарных электрических зарядов начального и конечного состояния.
Известно, суммарный электрический заряд начального состояния равен сумме зарядов ионов 16-О(8) и протонов водорода-мишени (1), то и сумма зарядов фрагментов должна равняться 9. В другим случае первичный ион не идентифицировался как 16-О и исключался из дальнейшей обработки. Хорошо этот отбор срабатывал для
событий без многозарядных (Z$_{f}$ > 3) фрагментов или при регистрации их вторичного взаимодействия. В общем случае в зарядовые распределения фрагментов вводились соответствующие поправки [100; С.192-196.]. Из этого следует, что, для каждого события определялись заряд Z$_{f}$ и полное число
фрагментов n$_{f}$ , полная множественность вторичных заряженных частиц, множественность отрицательно заряженных частиц, множественность положительно заряженных частиц, число V$_{0}$ (вилок
нейтральных частиц), число Далитц-пар (е$^{+}$ - е$^{-}$ -пар).

\newpage

\addcontentsline{toc}{section}{Заключение}
\section*{Заключение}




	\newpage
\begin{thebibliography}{57}
\bibitem{1}{N. P. Andreeva et al., Eur. Phys. Jour. A27, s1, 295 (2006).}
\bibitem{2} {Н. Г. Пересадько и др. ЯФ 70, 1266 (2007).}
\bibitem{3} {Д. А. Артеменков и др., ЯФ 70, 1261 (2007).}
\bibitem{4} {Т. В. Щедрина и др., ЯФ 70, 1271 (2007).}
\bibitem{5} {М. Карабова и др. ЯФ 72, 329 (2009).}
\bibitem{6} {П. И. Зарубин, Сообщение ОИЯИ No Р1-2010-75 (Дубна, 2010).}
\bibitem{7} {К.Н. Абдуллаева и др., ДАН РУз No 5, 21 (1996).}
\bibitem{8} {The BECQUEREL Project http://becquerel.jinr.ru}
\bibitem{9} {V. V. Belaga et al., Phys. Atom. Nucl. 58, 1905 (1995);arXiv:1109.0817}
\bibitem{10} {N. P. Andreeva et al., Phys. Atom. Nucl. 59, 102 (1996);arXiv:1109.3007}
\bibitem{11} {A. El-Naghy et al., J. Phys. G 14, 1125 (1988).}
\bibitem{12} {M. I. Adamovich et al., Phys. At. Nucl. 62, 1378 (1999);arXiv:1109.6422}
\bibitem{13} {M. I. Adamovich et al., J. Phys. G 30, 1479 (2004)}
\bibitem{14} {M. I. Adamovich et al., Part. Nucl. Lett. 110, 29 (2002);nucl-ex/0206013}
\bibitem{15} {M. I. Adamovich et al., Phys. At. Nucl. 67, 514 (2004);arXiv:nucl-ex/0301003}
\bibitem{16} {N. G. Peresadko et al., Phys. Atom. Nucl. 70, 1266 (2007); nucl-ex/0605014}
\bibitem{17} {T. V. Shchedrina et al., Phys. Atom. Nucl. 70, 1230 (2007);arXiv:nucl-ex/0605022}
\bibitem{18} {D. A. Artemenkov et al., Phys. Atom. Nucl. 70, 1226 (2007); nucl-ex/0605018}
\bibitem{19} {D. A. Artemenkov et al., Few Body Syst. 44, 273 (2008)}
\bibitem{20} {Karabova et al., Phys. Atom. Nucl. 72, 300(2009);arXiv:nucl-ex/0610023}
\bibitem{21} {R. Stanoeva et al., Phys. Atom. Nucl. 72, 690 (2009);arXiv:0906.4220}
\bibitem{22} {D. O. Krivenkov et al., Phys. Atom. Nucl. 73, 2103 (2010);arXiv:1104.2439}
\bibitem{23} {D. A. Artemenkov et al., Few Body Syst. 50, 259 (2011);arXiv:1105.2374}
\bibitem{24} {D. A. Artemenkov et al., Int. J. Mod. Phys. E 20, 993 (2011) arXiv:1106.1749}
\bibitem{25} {R. R. Kattabekov, K. Z. Mamatkulov et al., Phys. Atom. Nucl. 73, 2110 (2010); arXiv:1104.5320}
\bibitem{26} {R. R. Kattabekov et al., Phys. Atom. Nucl. to be published}
\bibitem{27} {K. Z. Mamatkulov et al., Phys. Atom. Nucl. to be published}
\bibitem{28} {H. L. Bradt and B. Peters, Phys. Rev. 77, 54 (1950)}
\bibitem{29} {C. F. Powell, P. H. Fowler, and D. H. Perkins, "The Study of Elementary Particles by the Photographic Method" Pergamon Press (1959)}
\bibitem{30} {A. M. Baldin, L. A. Didenko, Fortsch. Phys. 38, 261 (1990)}
\bibitem{31} {P. A. Rukoyatkin et al., EPJ ST 162, 267 (2008)}
\bibitem{32} {V. S. Barashenkov et al., Nucl. Phys. 9, 77 (1958/59)}
\bibitem{33} {M. G. Antonova et al., Phys. Lett. B 39, 285 (1972)}
\bibitem{34} {W. H. Barkas «Nuclear research emulsions» Academic Press (1963)}
\bibitem{35} {H. H. Heckman, D. E. Greiner, P. J. Lindstrom, and H. Shwe, Phys. Rev. C 17, 173 (1978)}
\bibitem{36} {E. M. Friedlander, H. H. Heckman, and Y. J. Karant, Phys. Rev. C 27, 2436 (1983)}
\bibitem{37} {P. L. Jain et al., Phys. Rev. 52, 1763 (1984)}
\bibitem{38} {G. Singh et al., Phys. Rev. C 41, 999 (1990)}
\bibitem{39} {G. Singh, P. L. Jain, Z. Phys. A 344,73 (1992)}
\bibitem{40} {G. Baroni et al., Nucl. Phys. A 516, 673 (1990)}
\bibitem{41} {G. Baroni et al. Nucl. Phys. A 540, 646 (1992)}
\bibitem{42} {M. I. Adamovich et al. Nucl. Phys. A 351, 311}
\bibitem{43} {M. I. Adamovich et al. Z. Phys. A, 359, 277 (1997)}
\bibitem{44} {M. I. Cherry et al., Eur. Phys. J. C 5, 641 (1998)}
\bibitem{45} {M. I. Adamovich et al., Eur. Phys. J. A 5, 429 (1999)}
\bibitem{46} {N. P. Andreeva et al., Phys. Atom. Nucl. 68, 455 (2005);arXiv:nucl-ex/0605015}
\bibitem{47} {N. G. Peresadko, V. N. Fetisov et al., JETP Lett. 88, 75 (2008);arXiv:1110.2881}
\bibitem{48} {Y. L. Parfenova and Ch. Leclercq-Willain, Phys. Rev. C 72, 054304 (2005)}
\bibitem{49} {Y. L. Parfenova and Ch. Leclercq-Willain, Phys. Rev. C 72, 024312(2005)}
\bibitem{50} {H. Esbensen and K. Hencken, Phys. Rev. C 61, 054606 (2000)}
\bibitem{51} {D. A. Artemenkov et al., J. Phys. Conf. Series 337, 012019 (2012)}
\bibitem{52} {H. Feshbach and K. Huang, Phys. Lett. B 47, 300 (1973)}
\bibitem{53} {A. S. Goldhaber, Phys. Lett. B 53, 306 (1974)}
\bibitem{54} {T. Toshito et al., Phys. Rev. C 78, 067602 (2008)}
\bibitem{55} {T. Aumann, Eur. Phys. J. A 26, 441 (2005)}
\bibitem{56}{ M. S. Swami, J. Schneps, and W. F. Fry, Phys. Rev. 103, 1134(1956).}
\bibitem{57}{S. Hyldegaard et al. Phys. Lett. B 678, 459 (2009)}
\end{thebibliography}
	
\end{document}
